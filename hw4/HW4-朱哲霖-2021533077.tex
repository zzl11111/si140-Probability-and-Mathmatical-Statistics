\documentclass{article}

\usepackage{fancyhdr}
\usepackage{extramarks}
\usepackage{amsmath}
\usepackage{amsthm}
\usepackage{amsfonts}
\usepackage{tikz}
\usepackage[plain]{algorithm}
\usepackage{algpseudocode}
\usepackage{enumerate}
\usepackage{tikz}
\usepackage{xifthen}
\usepackage{xparse}
\usepackage{amsmath, amssymb}
\usepackage{lipsum}
\usepackage{graphicx}
\usepackage{pythonhighlight}
\usetikzlibrary{automata,positioning}

%
% Basic Document Settings
%  

\topmargin=-0.45in
\evensidemargin=0in
\oddsidemargin=0in
\textwidth=6.5in
\textheight=9.0in
\headsep=0.25in

\linespread{1.1}

\pagestyle{fancy}
\lhead{\hmwkAuthorName}
\chead{\hmwkClass: \hmwkTitle}
\rhead{\firstxmark}
\lfoot{\lastxmark}
\cfoot{\thepage}

\renewcommand\headrulewidth{0.4pt}
\renewcommand\footrulewidth{0.4pt}

\setlength\parindent{0pt}

%
% Create Problem Sections
%

\newcommand{\enterProblemHeader}[1]{
    \nobreak\extramarks{}{Problem \arabic{#1} continued on next page\ldots}\nobreak{}
    \nobreak\extramarks{Problem \arabic{#1} (continued)}{Problem \arabic{#1} continued on next page\ldots}\nobreak{}
}

\newcommand{\exitProblemHeader}[1]{
    \nobreak\extramarks{Problem \arabic{#1} (continued)}{Problem \arabic{#1} continued on next page\ldots}\nobreak{}
    \stepcounter{#1}
    \nobreak\extramarks{Problem \arabic{#1}}{}\nobreak{}
}

\newcommand*\circled[1]{\tikz[baseline=(char.base)]{
		\node[shape=circle,draw,inner sep=2pt] (char) {#1};}}


\setcounter{secnumdepth}{0}
\newcounter{partCounter}
\newcounter{homeworkProblemCounter}
\setcounter{homeworkProblemCounter}{1}
\nobreak\extramarks{Problem \arabic{homeworkProblemCounter}}{}\nobreak{}

%
% Homework Problem Environment
%
% This environment takes an optional argument. When given, it will adjust the
% problem counter. This is useful for when the problems given for your
% assignment aren't sequential. See the last 3 problems of this template for an
% example.
%

\NewDocumentEnvironment{homeworkProblem}{s m}{
    \IfBooleanT{#1}{\newpage}
    \section{Problem \arabic{homeworkProblemCounter} {\small (#2)}}
    \setcounter{partCounter}{1}
    \enterProblemHeader{homeworkProblemCounter}

}{
    \exitProblemHeader{homeworkProblemCounter}
}

%
% Homework Details
%   - Title
%   - Due date
%   - Class
%   - Instructor
%   - Class number
%   - Name
%   - Student ID

\newcommand{\hmwkTitle}{Homework\ \#4}
\newcommand{\hmwkDueDate}{October 22, 2022}
\newcommand{\hmwkClass}{Probability and Mathematical Statistics}
\newcommand{\hmwkClassInstructor}{Professor Ziyu Shao}

\newcommand{\hmwkClassID}{\circled{0}}

\newcommand{\hmwkAuthorName}{Zhu Zhelin}
\newcommand{\hmwkAuthorID}{2021533077}

%
% Title Page
%

\title{
    \vspace{2in}
    \textmd{\textbf{\hmwkClass:\\  \hmwkTitle}}\\
    \normalsize\vspace{0.1in}\small{Due\ on\ \hmwkDueDate\ at 11:59am}\\
   \vspace{2in}\Huge{\hmwkClassID}\\   
   \vspace{2in}
}

\author{
	Name: \textbf{\hmwkAuthorName} \\
	Student ID: \hmwkAuthorID}
\date{}


\renewcommand{\part}[1]{\textbf{\large Part (\alph{partCounter})}\stepcounter{partCounter}\\}

%
% Various Helper Commands
%

% Useful for algorithms
\newcommand{\alg}[1]{\textsc{\bfseries \footnotesize #1}}
% For derivatives
\newcommand{\deriv}[1]{\frac{\mathrm{d}}{\mathrm{d}x} (#1)}
% For partial derivatives
\newcommand{\pderiv}[2]{\frac{\partial}{\partial #1} (#2)}
% Integral dx
\newcommand{\dx}{\mathrm{d}x}
% Alias for the Solution section header
\newcommand{\solution}{\textbf{\Large Solution}}
% Probability commands: Expectation, Variance, Covariance, Bias
\newcommand{\E}{\mathrm{E}}
\newcommand{\Var}{\mathrm{Var}}
\newcommand{\Cov}{\mathrm{Cov}}
\newcommand{\Bias}{\mathrm{Bias}}

\begin{document}

\maketitle
\pagebreak

% Problem 1
\begin{homeworkProblem}{{\color{blue}mention the source of question}, \textit{e.g.}, BH CH0 \#1}
	\solution
\\	\large{
First I want to calculate E(X)
$$E(X)=\sum_{k=1}^{\infty}\frac{cp^k}{k}\cdot k=\sum_{k=1}^{\infty}cp^k =\frac{cp}{1-p}$$\\
to calculate $Var(X)$,we can calculate $E(X^2)-(EX)^2$
$$E(X^2)=\sum_{k=1}^{\infty}\frac{cp^{k}}{k}\cdot k^2=c\sum_{k=1}^{\infty}kp^k=cp\sum_{k=1}^{\infty}\frac{\mathrm{d}p^k}{\mathrm{d}p}=cp\frac{\mathrm{d}\sum_{k=1}^{\infty}p^k}{\mathrm{d}p}=cp\cdot\frac{\mathrm{d}\frac{p}{1-p}}{\mathrm{d}p}=\frac{cp}{1-p^2}$$,
so the answer is $\frac{cp(1-cp)}{(1-p)^2}$
	}
\end{homeworkProblem}

% Problem 2
\begin{homeworkProblem}*{BH CH0 \#2}
\solution\\
\begin{enumerate}[(a)]
\item We can let $I_i$ be the indicator random variable  represent that in the $i_{th}$ trial,show if two people simultanously success,$P(I_i)=p_1p_2,P(I_i^c)=1-p_1p_2$,
and $I_i$ is $i.i.d$ satisfy $I_i\sim Bern(p_1p_2)$,then $I$ satisfy $Bin(n,p_1p_2)$ ,then I can define $Z\sim Geom(p_1p_2)$ represent the distribution of the Bern trials,Z is the number of faiure trial number before first success happen,then $Y=Z+1$ satisfy the distribution of first success happen,so the distribution is $P(Y=k)=P(Z+1=k)=P(Z=k-1)=(1-p_1p_2)^{k-1}p_1p_2$\\
the expected time is $E(Y)=E(Z+1)=E(Z)+1=\frac{1}{p_1p_2}$
\item the same as below,we can define $I_i$ as at least one success happend at $i_{th}$ trial ,$P(I_i^c)=(1-p_1)(1-p_2),P(I_i)=1-(1-p_1)(1-p_2)$,$I_i$satify $\sim$Bern$(1-(1-p_1)(1-p_2))$4then the number of trials(including first success) until at least one success happen represented by $Y$,then $Z=Y-1$ $\sim$ Geom$(1-(1-p_1)(1-p_2))$
$E(Y)=E(Z+1)=\frac{1}{1-(1-p_1)(1-p_2)}$
\item considering the random variable X be the time in $k_{th}$ trial,they simultaneously and firstly win,when it happens,it should satisfy that $(k-1)$trials they both failure the possibilty of both failure is $(1-p_1)(1-p_2)=(1-p_1)^2$,then P(X=K)=$(1-p_1)^{2(k-1)}p_1^2$ ,then sum all the circumstances up ,I can get that $\sum_{k=1}^{\infty}(1-p_1)^{2(k-1)}p_1^2=\frac{p_1}{2-p_1}$,then according to symmetry,the probability of Nick's first success precedes Penny is equal to Penny's first success preccedes Nick's success ,we can get the probability by calculate $\frac{1}{2}(1-\frac{p_1}{2-p_1})=\frac{1-p_1}{2-p_1}$
\end{enumerate}


\end{homeworkProblem}


\begin{homeworkProblem}*{BH CH0 \#3}
	\begin{enumerate}[(a)]
	\item We can first define X as the different number of stops it will happen among k people and X$=I_2+I_3+\cdots I_n$ among them

	$I_j=1$(when $j_{th}$ floor stop),$I_j=$0,other,$E(I_j)=1-P$(nobody stop at $j_{th}$floor)=1-$\left(\frac{n-2}{n-1} \right)^k$
	$E(X)=E(I_2+I_3+\cdots I_n )=(n-1)\cdot(1-(1-\frac{1}{n-1})^k)$
\item the same as below ,but the $E(I_j)$ changes to be $1-(1-p_j)^k$,so the total probability is $\sum_{j=2}^{n}1-(1-p_j)^k=n-1-\sum_{j=2}^{n}(1-p_j)^k$
	\end{enumerate}
\end{homeworkProblem}
\begin{homeworkProblem}*{BH CH0 \#4}
	\solution
	\begin{enumerate}[(a)]
\item the LOTUS define that $E(g(x))=\sum_{x}g(x)P(X=x)$,to calculate the lefthandside I can let $h(X)=Xg(X)$,then $$E(h(X))=\sum\limits_{k}kg(k)P(X=k)=\sum\limits_{k=0}kg(k)\cdot \frac{e^{-\lambda}\lambda ^k}{k!}=\lambda \sum\limits_{k=1}^{\infty}g(k)\frac{e^{-\lambda}\lambda ^{k-1}}{(k-1)!}=\lambda\sum\limits_{k=0}^{\infty}g(k+1)\frac{e^{-\lambda}\lambda^{k}}{k!}$$
and the rhs $$\lambda E(g(X+1))=\lambda \sum\limits_{k=0}^{\infty}g(k+1)\frac{e^{-\lambda}\lambda^{k}}{k!}=lhs$$,so Q.E.D
\item $$E(X^3)=E(X\cdot X^2)=\lambda E((X+1)^2)=\lambda E(X^2+2X+1)=\lambda(E(X^2)+2E(X)+E(1))=\lambda(E(X\cdot X)+2E(X\cdot 1)+1)=$$\\ $$\lambda(\lambda E(X+1)+2\lambda E(1)+1)=\lambda^3+3\lambda^2+\lambda$$
	\end{enumerate}		

\end{homeworkProblem}

\begin{homeworkProblem}*{BH CH0 \#5}
	\begin{enumerate}[(a)]
\item First ,I will use the property of CDF that is if$m_1\leq m_2$,then $P(Y\leq m_1)\leq P(Y\leq m_2)$
since$P(Y\leq 23)=0.507\geq 0.5,P(Y\geq 23)=1-P(Y\leq 22)$,$P(Y\leq 22)<0.5$,so $P(Y\geq 23)\geq 0.5$,so 23 is the median,then to prove the uniquness,if $m>23,P(Y\geq m)=1-P(Y\leq m-1)\leq 0.493$ doesn't satisfy $P(Y\geq m)\geq 0.5$,then if $m<23$,P$(Y\leq m)\leq P(Y\leq 22)<0.5$,which also doesn't satisfy,so that 23 is the only median.
\item To show the fact,we can first consider when X=j,since $X\geq j$then $I_1\cdots I_j=1$,$I_{j+1}\cdots I_{366}=0$,$X=I_1+I_2+\cdots I_{366}$,then I want to deduce from right to left.I can define the max index j which satisfy $I_j=1$ ,it indicates $X\leq j$,then all the variable $I_k,k\leq j$ must be 1 since $j\geq k$,since j is the max index ,for any $i>j$,$I_i=0$,then we can get that $X=j$ exactly ,so max index j $I_j=1\Rightarrow I_1+I_2+\cdots I_{366} =j$ and $X=j$,it is a bijection,thus prove the equality
$$E(X)=E(I_1)+E(I_2)+\cdots E(I_{366})=\sum\limits_{j=1}^{366} p_j$$
\item use python to solve it numerically,we can get the expected time equal to $24.616$
\item We want to calculate $Var(X)$,$Var(X)=E(X^2)-(EX)^2$,$E(X^2)$,first,I want to handle $X^2=I_1^2+I_2^2\cdots I_{366}^2+2\sum\limits_{j=2}^{366}\sum\limits_{i=1}^{j-1}I_iI_j$,then $I_k^2=I_k$(proved by enumerate two circumstances),$I_iI_j$when(i$<j$),this equalt to $I_j^2=I_j$,so it can be reduced to$I_1+\cdots I_{366}+2\sum\limits_{j=2}^{366}(j-1)I_j$,so E($X^2$)=$E(X)+2\sum\limits_{j=2}^{366}(j-1)p_j$,Var(X)=$\sum\limits_{j=1}^{366}p_j-(\sum\limits_{j=1}^{366}p_j)^2+2\sum\limits_{j=2}^{366}(j-1)p_j$ using python I can calculate it numerically as 148.64

\end{enumerate}
\begin{python}
sum=2
p=1
for  i in range(3,367):
 p=p*(1-(i-2)/365)
 sum+=p
s=0 
p=1
for i in range(2,367):
    p=p*(1-(i-2)/365)
    s+=2*(i-1)*p
v=sum-sum*sum+s
print(v)
\end{python}
\end{homeworkProblem}

\end{document}
