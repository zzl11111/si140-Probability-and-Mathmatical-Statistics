\documentclass{article}

\usepackage{fancyhdr}
\usepackage{extramarks}
\usepackage{amsmath}
\usepackage{amsthm}
\usepackage{amsfonts}
\usepackage{tikz}
\usepackage[plain]{algorithm}
\usepackage{algpseudocode}
\usepackage{enumerate}
\usepackage{tikz}
\usepackage{xifthen}
\usepackage{xparse}
\usepackage{amsmath, amssymb}
\usepackage{lipsum}

\usetikzlibrary{automata,positioning}

%
% Basic Document Settings
%  

\topmargin=-0.45in
\evensidemargin=0in
\oddsidemargin=0in
\textwidth=6.5in
\textheight=9.0in
\headsep=0.25in

\linespread{1.1}

\pagestyle{fancy}
\lhead{\hmwkAuthorName}
\chead{\hmwkClass: \hmwkTitle}
\rhead{\firstxmark}
\lfoot{\lastxmark}
\cfoot{\thepage}

\renewcommand\headrulewidth{0.4pt}
\renewcommand\footrulewidth{0.4pt}

\setlength\parindent{0pt}

%
% Create Problem Sections
%

\newcommand{\enterProblemHeader}[1]{
    \nobreak\extramarks{}{Problem \arabic{#1} continued on next page\ldots}\nobreak{}
    \nobreak\extramarks{Problem \arabic{#1} (continued)}{Problem \arabic{#1} continued on next page\ldots}\nobreak{}
}

\newcommand{\exitProblemHeader}[1]{
    \nobreak\extramarks{Problem \arabic{#1} (continued)}{Problem \arabic{#1} continued on next page\ldots}\nobreak{}
    \stepcounter{#1}
    \nobreak\extramarks{Problem \arabic{#1}}{}\nobreak{}
}

\newcommand*\circled[1]{\tikz[baseline=(char.base)]{
		\node[shape=circle,draw,inner sep=2pt] (char) {#1};}}


\setcounter{secnumdepth}{0}
\newcounter{partCounter}
\newcounter{homeworkProblemCounter}
\setcounter{homeworkProblemCounter}{1}
\nobreak\extramarks{Problem \arabic{homeworkProblemCounter}}{}\nobreak{}

%
% Homework Problem Environment
%
% This environment takes an optional argument. When given, it will adjust the
% problem counter. This is useful for when the problems given for your
% assignment aren't sequential. See the last 3 problems of this template for an
% example.
%

\NewDocumentEnvironment{homeworkProblem}{s m}{
    \IfBooleanT{#1}{\newpage}
    \section{Problem \arabic{homeworkProblemCounter} {\small (#2)}}
    \setcounter{partCounter}{1}
    \enterProblemHeader{homeworkProblemCounter}

}{
    \exitProblemHeader{homeworkProblemCounter}
}

%
% Homework Details
%   - Title
%   - Due date
%   - Class
%   - Instructor
%   - Class number
%   - Name
%   - Student ID

\newcommand{\hmwkTitle}{Homework\ \#8}
\newcommand{\hmwkDueDate}{September 18, 2022}
\newcommand{\hmwkClass}{Probability and Mathematical Statistics}
\newcommand{\hmwkClassInstructor}{Professor Ziyu Shao}

\newcommand{\hmwkClassID}{\circled{0}}

\newcommand{\hmwkAuthorName}{Zhu Zhelin}
\newcommand{\hmwkAuthorID}{2021533077}

%
% Title Page
%

\title{
    \vspace{2in}
    \textmd{\textbf{\hmwkClass:\\  \hmwkTitle}}\\
    \normalsize\vspace{0.1in}\small{Due\ on\ \hmwkDueDate\ at 11:59am}\\
   \vspace{2in}\Huge{\hmwkClassID}\\   
   \vspace{2in}
}

\author{
	Name: \textbf{\hmwkAuthorName} \\
	Student ID: \hmwkAuthorID}
\date{}


\renewcommand{\part}[1]{\textbf{\large Part (\alph{partCounter})}\stepcounter{partCounter}\\}

%
% Various Helper Commands
%

% Useful for algorithms
\newcommand{\alg}[1]{\textsc{\bfseries \footnotesize #1}}
% For derivatives
\newcommand{\deriv}[1]{\frac{\mathrm{d}}{\mathrm{d}x} (#1)}
% For partial derivatives
\newcommand{\pderiv}[2]{\frac{\partial}{\partial #1} (#2)}
% Integral dx
\newcommand{\dx}{\mathrm{d}x}
% Alias for the Solution section header
\newcommand{\solution}{\textbf{\Large Solution}}
% Probability commands: Expectation, Variance, Covariance, Bias
\newcommand{\E}{\mathrm{E}}
\newcommand{\Var}{\mathrm{Var}}
\newcommand{\Cov}{\mathrm{Cov}}
\newcommand{\Bias}{\mathrm{Bias}}

\begin{document}

\maketitle
\pagebreak

% Problem 1
\begin{homeworkProblem}{{\color{blue}mention the source of question}, \textit{e.g.}, BH CH0 \#1}
	\begin{enumerate}[(a)]
\item the MGF of X is $E(e^{tX})=E(e^{t(I(A_1)+I(A_2)+I(A_3)+\cdots)})$ since all the events are independent this is equal to $\prod \limits_{i=0}^n E(e^{tI(A_i)})$ considering $A_i$ since it is a bernoulli trial , $E(e^{tI(A_i)})=p_i e^{t}+q_i$,so the total is $\prod \limits_{i=0}^n (p_i e^{t}+1-p_i)$
\item the MGF is equal to $\prod \limits_{i=1}^n(p_i(e^{t}-1)+1)$ using the approximation ,we can get this equal to $\prod \limits_{i=0}^n(e^{p_i(e^{t}-1)})$,since the total sum of $p_i=\lambda$,so this is equal to $e^{\lambda(e^{t}-1)}$ since the event can be an approximation of possion process, considering the true possion process $X\sim $ Pois($\lambda$)and its MGF $E(e^{tX})=\sum\limits_{k=0}^{\infty}e^{tk}\frac{e^{-\lambda}(\lambda)^{k}}{k!}=\sum\limits_{k=0}^{\infty}e^{tk}\frac{e^{-\lambda}(e^{t}\lambda)^k}{k!}=e^{\lambda(e^t-1)}$,since it is approximation about pois,so in the limit context the approximation is equal to pois. 
	\end{enumerate}
\end{homeworkProblem}

% Problem 2
\begin{homeworkProblem}*{BH CH0 \#2}
\begin{enumerate}[(a)]
\item  to calculate P(L=l,M=m),when $m<l$ the joint PMF is equal to 0,when $m=l,$ if and only if $X=Y=l$ the PMF=$q^{2l}p^2$,when $m>l$,the PMF is the sum of $P(X=l,Y=M)+P(X=m,Y=l)=2q^{l+m}p^2$ since given L=10 $P(M=5|L=10)=0,P(M=5)\neq 0$,so they are not independent.
\item $P(L=l)=\sum\limits_{k=l}^{\infty} P(L=l,M=k)=q^{2l}p^2+\sum\limits_{m=l+1}^{\infty}2q^{l+m}p^{2}=q^{2l}(p^2+2pq)$ ,using the story ,considering L=l,it means in the l trial before,both X and Y is failure $q^{2l}$,considering the l+1 trial there must be at least one success the both success $p^2$,only one success $2pq$ add them all and multiply the previous chance,we can get that$q^{2l}(p^2+2pq)$,on another way ,we can see at least one success as one bernoulli success,so the PMF of L it has $Bern(1-q^2)$ since $p^{2}+2pq+q^{2}=1,p^{2}+2pq=1-q^{2}$
\item since $L+M=X+Y$ $E(L+M)=E(X+Y)$$E(L)+E(M)=2E(X)=2\frac{q}{p}$,$E(L)=\sum\limits_{l=0}^{n}lq^{2l}(p^2+2pq)=\frac{q^2}{1-q^2}$,so $E(M)=\frac{q(1+2q)}{p(1+q
)} $ 
\item P(L=l,M-L=k) when k=0 the joint PMF is $q^{2l}p^2$,when k$>0$,then PMF is $P(X=l,Y=l+k)+P(X=l+k,Y=l)=2q^{2l+k}p^2$ ,it can be divided into $f(l)g(k)$ where P$(L=l)=q^{2l}(p^2+2pq)$ g(k)=$\frac{2p^2q^{k}}{p^2+2pq}(k>0)$,$\frac{p^2}{p^2+2pq}(k=0)$ since it can be divided into two function $f(l)g(k)$,it is independent
\end{enumerate}
\end{homeworkProblem}


\begin{homeworkProblem}*{BH CH0 \#3}
\begin{enumerate}[(a)]
	\item P(T$\leq t|X=x$)=$P(Y\leq t-x)$ since Y$\sim Expo(\lambda)$,the CDF is $1-e^{-\lambda(t-x)} $$(t> x)$,CDF is 0$t\leq x$
	\item we can easily get this by differntiating $f_{T|X}(t|X=x)=\lambda e^{-\lambda(t-x)}$ (when t$\geq x$),else(0),to check the validation,the PDF is obviously larger than 0,then $\int _{x}{\infty}\lambda e^{-\lambda(t-x)}dt=1$
\item let $t>0$ $f_{X|T}(x|t)=\frac{f_{T|X}(t|x)f_X(x)}{f_{T}(t)}\propto f_{T|X}(t|x)f_{X}(x)=\lambda^{2}e^{-\lambda t} $ it is a constant so the event $X|T\sim Unif(0,T)$ so the conditional PDF of X is $\frac{1}{t}$
\item $f_T(t)=\frac{f_{T|X}(t|x)f_{X}(x)}{f{X|T}(x|t)}=\frac{\lambda^2e^{-\lambda t}}{\frac{1}{t}}=\lambda^{2}te^{-\lambda t}$
\end{enumerate}
\end{homeworkProblem}

\begin{homeworkProblem}*{BH CH0 \#4}
\begin{enumerate}[(a)]
\item the marginal CDF is $P(M\leq m)=P(U_1\leq m,U_2\leq m,U_3\leq m)$since the random variable is independent =$m^{3}$ so PDF is $3m^2$,$P(M\leq m)=P(L\leq l,M\leq m)+P(L\geq l,M\geq m)$ $P(L\geq l,M\leq m)=P(l\leq U_1\leq m,l\leq U_2\leq m,l\leq U_3\leq m)$ when $l>m$,this is equal to 0,so the joint CDF is $m^{3}-(m-l)^{3}$,then differntiating it we can get the joint PDF is $6(m-l)$
\item The marginal PDF of $f_{L}(l)=3(1-l)^2$ it is the sum of $(P(U_i=l,U_j\geq l,U_k\geq l))$,$f_{M|L}(m|l)=\frac{f(m,l)}{f_{L}(l)}=\frac{2(m-l)}{(1-l)^2}$
\end{enumerate}
\end{homeworkProblem}
\begin{homeworkProblem}*{BH CH0\#5}
    \begin{enumerate}[(a)]
        \item let q=1-p,let X be the trial vector$\sim Mult_{k}(n,(p^2,2pq,q^2)) $ $P(X_1=x_1,X_2=x_2,X_3=x_3)=\frac{n!}{x_1!x_2!x_3!}p^{2x_1}(2pq)^{x_2}q^{2x_3}$
        \item We can see it as the Bernoulli trial $\sim(1-q^2)$ when choosing a person,if he is $aa$ then failure,otherwise success,so the distribution of number in the sample who have an A$\sim Bin(n,1-q^2)$
        \item  it can be seen as 2n Bernoulli trials when assign A success,assign a failure,so for the total genes ,let Y be the genes contain A$Y\sim Bin(2n,p)$
        \item $x_1,x_2,x_3$ be the observed data,according to (a) the possibility is $\propto p^{2x_1}(pq)^{x_2}q^{2x_3}$,let $L(p)=p^{2x_1}(pq)^{x_2}q^{2x_3}$to maximum the possibility,we can take the log for the both side and take the differntiating ,let it to be 0,$logL(p)=(2x_1+x_2)logp+(x_2+2x_3)log(1-p)$ let its derivative to be zero,I can get that $p=\frac{2x_1+x_2}{2n}$
        \item  We can see it as many times of Bernoulli trials ,the observed data Y indicate how many aa are found$\sim Bin(n,q^2)$ so to find $L(q)$ we want to maximum$q^{2y}(1-q^2)^{n-y}$ using the same technique,we can get $log(L(q))=2yLog(q)+(n-y)log(1-q^2)$ then to calculate the derivative,let it equal to 0,we can get $q=\sqrt{\frac{y}{n}}$ so p=$1-\sqrt{\frac{y}{n}}$
    \end{enumerate}
\end{homeworkProblem}
\end{document}
